\documentclass[11pt]{article}
\usepackage[top=2.1cm,bottom=2cm,left=2cm,right= 2cm]{geometry}
%\geometry{landscape}                % Activate for for rotated page geometry
\usepackage[parfill]{parskip}    % Activate to begin paragraphs with an empty line rather than an indent
\usepackage{graphicx}
\usepackage{amssymb}
\usepackage{epstopdf}
\usepackage{amsmath}
\usepackage{multirow}
\usepackage{hyperref}
\usepackage{changepage}
\usepackage{lscape}
\usepackage{ulem}
\usepackage{multicol}
\usepackage{dashrule}
\usepackage[usenames,dvipsnames]{color}
\usepackage{enumerate}
\usepackage{amsmath}
\newenvironment{rcases}
  {\left.\begin{aligned}}
  {\end{aligned}\right\rbrace}

\newcommand{\urlwofont}[1]{\urlstyle{same}\url{#1}}
\newcommand{\degree}{\ensuremath{^\circ}}
\newcommand{\hl}[1]{\textbf{\underline{#1}}}



\DeclareGraphicsRule{.tif}{png}{.png}{`convert #1 `dirname #1`/`basename #1 .tif`.png}

\newenvironment{choices}{
\begin{enumerate}[(a)]
}{\end{enumerate}}

%\newcommand{\soln}[1]{\textcolor{MidnightBlue}{\textit{#1}}}	% delete #1 to get rid of solutions for handouts
\newcommand{\soln}[1]{ \vspace{1.35cm} }

%\newcommand{\solnMult}[1]{\textbf{\textcolor{MidnightBlue}{\textit{#1}}}}	% uncomment for solutions
\newcommand{\solnMult}[1]{ #1 }	% uncomment for handouts

%\newcommand{\pts}[1]{ \textbf{{\footnotesize \textcolor{black}{(#1)}}} }	% uncomment for handouts
\newcommand{\pts}[1]{ \textbf{{\footnotesize \textcolor{blue}{(#1)}}} }	% uncomment for handouts

\newcommand{\note}[1]{ \textbf{\textcolor{red}{[#1]}} }	% uncomment for handouts

\begin{document}


\enlargethispage{\baselineskip}

Fall 2019, MATH 347 \hfill Jingchen (Monika) Hu\\

\begin{center}
{\huge Case Study 2: True-or-False Exam Scores}	
\end{center}
\vspace{0.5cm}


\section{The data} 

Suppose a group of 15 people sit an exam made up of 40 true-or-false questions, and each receives a score afterwards. Dataset ``TrueFalseScores.csv" on Moodle.

\section{Some questions} 

\begin{itemize}
\item[1.] Some analyses claim that these scores suggest that the first 5 people were just guessing, but the last 10 had some level of knowledge. What do you think?

\item[2.] What are your Bayesian methods? What assumptions are you making? Are the assumptions justified?\\

Possible direction: 1) people come from two groups (i.e. guessing group vs knowledge group), and these groups have different probabilities of success; 2) one step further is to consider whether people have individual-specific success probability.

\item[3.] What are the parameter(s) in your model(s)? What prior distribution(s) do you use? Are full conditional distribution(s) recognizable?

\item[4.] What is your computation scheme? Some useful JAGS function:
\begin{itemize}
\item[(a)] equals(x, y): test for equality (logical)
\item[(b)] T(a, b): truncate $[a, b]$.\\
 e.g. dbeta(1, 1)T(0.5,1) assigns a Uniform distribution on $[0.5, 1]$.\\
 (recall that Uniform(0,1)=Beta(1,1))
\end{itemize}

\item[5.] What's your posterior summary of your parameter(s)? (Don't forget MCMC diagnostics to make sure your chain has converged.)

\item[6.] What is your conclusion?
\end{itemize}










\end{document} 